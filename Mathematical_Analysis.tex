\documentclass[9pt]{ctexart}
\CTEXsetup[format={\Large\bf}]{section}
\usepackage{lmodern}
\usepackage{geometry}
\usepackage{titlesec}
\usepackage{amsmath}
\usepackage{fancyhdr}
\pagestyle{plain}
\usepackage{amssymb}
\usepackage{bm}
\usepackage{graphicx}
\usepackage{braket}
\usepackage{hyperref}
\usepackage{ntheorem}
\geometry{top=2.5cm,bottom=3cm}
\theorembodyfont{\normalfont}

\newtheorem{definition}{Definition}[section]
\newtheorem{theorem}{Theorem}[section]
\title{\bf{数学分析:提纲与注解}}
\author{\textsc{Frederick Wang}}
\date{}
\begin{document}
\maketitle
\section{集合与映射}
\par{基础概念与定义略去.}
\subsection{集合}
\begin{theorem}
    可列个可列集之并也是可列集.
\end{theorem}

\par{事实上,用方块阵将可列集排列起来,再按对角线法则排列成一个新的可列集即可.}

\begin{theorem}
    有理数集$\mathbb{Q}$是可列集.
\end{theorem}

\subsection{映射与函数}

\begin{theorem}
    \textbf{(三角不等式)}对于任意实数$a$和$b$,都有
    \[||a|-|b||\leq |a+b|\leq |a|+|b|\]
\end{theorem}

\begin{theorem}
    \textbf{(均值不等式)}对任意$n$个正数$a_1,...,a_n$,有
    \[\frac{\sum_{i=1}^n a_i}{n}\leq \sqrt[n]{\prod_{i=1}^n a_i}\leq n/(\sum_{i=1}^n \frac{1}{a_i})\]
\end{theorem}

\part{极限理论}
\section{数列极限}
\subsection{实数系的连续性}
\begin{theorem}
    \textbf{(确界存在与唯一性)}非空有上界的数集必有上确界,非空有下界的数集必有下确界,且确界的存在是唯一的.
\end{theorem}
\subsection{数列极限}
\begin{definition}
    设$\{x_n\}$是一给定数列,$a$是一个实常数,如果对于任意给定的$\epsilon>0$,可以找到正整数$n$,使得当$n>N$时,成立
    \[|x_n-a|<\epsilon\]
    则称数列$\{x_n\}$收敛于$a$.
\end{definition}

\begin{theorem}
    数列极限的性质:
    \begin{itemize}
        \item[(1)] 收敛数列的极限必唯一.
        \item[(2)] 收敛数列必然有界.
        \item[(3)] 设数列$\{x_n\}$,$\{y_n\}$均收敛,若$\lim_{n\to\infty}x_n=a$, $\lim_{n\to\infty}y_n=b$,且$a<b$,则存在正整数$n$,当$n>N$时,成立$x_n<y_n$.
        \item[(4)] 夹逼性.
        \item[(5)] 极限的四则运算.(注意:仅仅对于有限次的四则运算才是正确的)
    \end{itemize}
\end{theorem}
\subsection{无穷大量}
\begin{definition}
    若对于任意给定的$G>0$,可以找到正整数$N$,使得当$n>N$时,成立
    \[|x_n|>G\]
    则称数列$\{x_n\}$为无穷大量,记作$\lim_{n\to\infty}=\infty$.\\
    同理可以定义定号的无穷大量,只需要从某一项开始都为正或负,就为正(或负)无穷大量.
\end{definition}
\par{一些Trivial的性质不再赘述,实际使用也很容易证明.}
\begin{theorem}
    \textbf{(Stolz定理)}设$y_n$是\textbf{严格单调增加}的\textbf{正无穷大量},且
    \[\lim_{n\to\infty}\frac{x_n-x_{n-1}}{y_n-y_{n-1}}=a\]
    其中$a$可以为有限量或者正负无穷,则
    \[\lim_{n\to\infty}\frac{x_n}{y_n}=a.\]
\end{theorem}

\par{一定注意Stolz定理使用的范围,分母要是严格单调递增的正无穷大量。主要用于各种无限次四则运算处理起来比较麻烦的极限.}
\subsection{收敛准则}
\begin{theorem}
    单调有界数列必定收敛.
\end{theorem}
\par{这一性质可以让我们先判断敛散性再去求具体的极限,而不是反过来.在求极限的时候仍然要切记极限的四则运算规则.}
\begin{definition}
    如果一列闭区间$\{[a_n,b_n]\}$满足条件:
    \begin{itemize}
        \item[(1)] $[a_{n+1},b_{n+1}]\subset[a_n,b_n]$,$n=1,2,3...$
        \item[(2)] $\lim_{n\to\infty}(b_n-a_n)=0$
    \end{itemize}
    则称这列闭区间形成一个闭区间套.
\end{definition}

\begin{theorem}
    \textbf{(闭区间套定理)}如果$\{[a_n,b_n]\}$形成一个闭区间套,则存在唯一的实数$\xi$属于所有的闭区间$[a_n,b_n]$,且$\xi=\lim_{n\to\infty}a_n=\lim_{n\to\infty}b_n$.
\end{theorem}

\begin{definition}
    设$\{x_n\}$是一个数列,而
    \[n_1<n_2<...<n_k<n_{k+1}<...\]
    是一列严格单调增加的正整数,则
    \[x_{n_1},...,x_{n_k},...\]
    也形成一个数列,成为数列$\{x_n\}$的子列.
\end{definition}

\begin{theorem}
    若数列$\{x_n\}$收敛于$a$,则它的任何子列$\{x_{n_k}\}$也收敛于$a$.反之,若存在数列$\{x_n\}$的两个子列分别收敛于不同的极限,则数列$\{x_n\}$必然发散.
\end{theorem}

\begin{theorem}
   \textbf{(Weierstrass定理)}有界数列必有收敛子列.\\相似地,若$\{x_n\}$是一个无界数列,则存在子列$\{x_{n_k}\}$,使得$\lim_{n\to\infty}=\infty$.  
\end{theorem}

\begin{definition}
    如果数列$\{x_n\}$具有以下特性:对于任意给定的$\epsilon>0$,存在正整数$N$,使得当$n,m>N$时,成立
    \[|x_n-x_m|<\epsilon\]
    则称数列$\{x_n\}$是一个基本数列.
\end{definition}

\begin{theorem}
    \textbf{(Cauchy收敛原理)}数列$\{x_n\}$收敛的充分必要条件是:$\{x_n\}$是基本数列.
\end{theorem}

\begin{theorem}
    \textbf{(有限覆盖定理)}$A$是一个开区间,若$A$覆盖$[a,b]$,则存在$A$中有限个开区间可以覆盖$[a,b]$.
\end{theorem}
\begin{theorem}
    实数系的基本定理(确界存在定理,单调有界数列收敛,闭区间套定理,Weierstrass定理,Cauchy收敛定理,有限覆盖定理)是等价的.
\end{theorem}


\section{函数极限与连续函数}
\subsection{函数极限}
\begin{definition}
    设函数$y=f(x)$在点$x_0$的某个去心邻域中有定义,如果存在实数$A$,对于任意给定的$\epsilon>0$,可以找到$\delta>0$,使得当$0<|x-x_0|<\delta$时,成立
    \[|f(x)-A|<\epsilon\]
    则称$A$是函数$f(x)$在点$x_0$的极限,记为
    \[\lim_{x\to x_0}f(x)=A.\]
    如果不存在具有上述性质的$A$,则称$f(x)$在点$x_0$的极限不存在.
\end{definition}
\par{函数极限的性质:(1)唯一性;(2)局部保序性(类比数列极限);(3)局部有界性(若极限存在,存在邻域使得函数值有界);(4)夹逼性.其四则运算性质与数列极限类似.}

\begin{theorem}
    \textbf{(Heine定理)} $\lim_{x\to x_0}f(x)=A$的充分必要条件是:对于任意满足条件的$\lim_{n\to\infty}f(x)=x_0$,且$x_n\neq x_0,\ \forall n>0$的数列$\{x_n\}$,相应的函数值数列$\{f(x_n)\}$成立
    \[\lim_{n\to\infty}f(x_n)=A.\]
\end{theorem}
\par{一般而言,构造两个具有相同极限的数列,然后证明其函数值极限不相同,是证明某个函数极限不存在的方法.事实上,Heine定理可以只关注极限的存在性.}
\begin{theorem}
    $\lim_{x\to x_0}f(x)$存在的充分必要条件是:对于任意满足条件的$\lim_{n\to\infty}f(x)=x_0$,且$x_n\neq x_0,\ \forall n>0$的数列$\{x_n\}$,相应的函数值数列$\{f(x_n)\}$收敛.
\end{theorem}
\par{对于极限的定义,实际上是可以进行扩充的.比如单侧极限的定义,或者也可以结合无穷大量的定义进行定义扩充,此处是trivial的,从略.并且同样的,对于函数而言,Cauchy收敛原理仍然成立.}
\begin{theorem}
    \textbf{(Cauchy收敛原理)}函数极限$\lim_{x\to\infty}f(x)$存在且有限的充分必要条件是:对于任意的$\epsilon>0$,存在$X>0$,使得对一切$x',x''>X$,成立
    \[|f(x')-f(x'')|<\epsilon\]
\end{theorem}
\par{同样可以写出$x$趋向于不同的值下的Cauchy收敛原理的形式,此处从略.}

\subsection{连续函数}
\begin{definition}
    设函数$f(x)$在点$x_0$的某个邻域中有定义,并且成立
    \[\lim_{x\to x_0}f(x)=f(x_0)\]
    则称函数$f(x)$在点$x_0$连续,$x_0$是函数$f(x)$的连续点.
\end{definition}

\begin{definition}
    若函数$f(x)$在区间$(a,b)$的每一点都连续,则称函数$f(x)$在开区间$(a,b)$上连续.
\end{definition}

\par{利用上一节定义的左极限和右极限,同样可以定义左连续和右连续.我们扩充了单侧连续的概念,从而有}
\begin{definition}
    若$f(x)$在$(a,b)$连续,且在左端点$a$右连续,右端点$b$左连续,则称函数$f(x)$在闭区间$[a,b]$上连续.
\end{definition}
\par{连续函数依据函数极限,仍然有四则运算的规律.接下来讨论不连续点的类型.}
\begin{itemize}
    \item[(1)] 第一类不连续点(跳跃点):函数$f(x)$在点$x_0$的左右极限都存在但不相等,即$f(x_0+)\neq f(x_0-)$.
    \item[(2)] 第二类不连续点:函数$f(x)$在点$x_0$的左、右极限至少有一个不存在.
    \item[(3)] 第三类不连续点(可去间断点):函数$f(x)$在点$x_0$的左右极限都存在且相等,但不等于$f(x_0)$或者$f(x)$在点$x_0$处无定义.
\end{itemize}

\begin{theorem}
    若函数$y=f(x),\ x\in D_f$是严格单调增加(减少)的,则存在它的反函数$x=f^{-1}(y),\ y\in R_f$,并且$f^{-1}(y)$也是严格单调增加(减少)的.
\end{theorem}

\begin{theorem}
    \textbf{(反函数连续性定理)}设函数$y=f(x)$在闭区间$[a,b]$上连续且严格单调增加,$f(a)=\alpha,\ f(b)=\beta$,则其反函数$x=f^{-1}(y)$在$[\alpha,\beta]$连续且严格单调增加.
\end{theorem}

\begin{theorem}
    若$u=g(x)$在点$x_0$连续,$g(x_0)=u_0$,又$y=f(u)$在点$u_0$连续,则复合函数$y=f\circ g(x)$在点$x_0$连续.
\end{theorem}
\par{事实上,由所有上述条件归纳出来,一切初等函数在其定义区间上连续.}
\subsection{无穷小量与无穷大量的阶}
\par{只需认识到由函数定义的无穷小量和无穷大量,了解阶、等价量的概念即可,此处从略.}
\subsection{闭区间上的连续函数}
\begin{theorem}
    若函数$f(x)$在闭区间$[a,b]$上连续,则:
    \begin{itemize}
        \item[(1)] $f(x)$在$[a,b]$上有界.
        \item[(2)] $f(x)$在$[a,b]$上能取到最大值和最小值.
        \item[(3)] 若$f(a)f(b)<0$,则一定存在$\xi\in(a,b)$,使得$f(\xi)=0$.
        \item[(4)] $f(x)$可以取到最大值$M$和最小值$m$之间的任意一个值.
    \end{itemize}
\end{theorem}

\begin{definition}
    设函数$f(x)$在区间$X$上定义,若对于任意给定的$\epsilon>0$,存在$\delta>0$,只要$x',x''\in X$满足$|x'-x''|<\delta$,就成立$|f(x')-f(x'')|<\epsilon$,则称$f(x)$在区间$X$上\textbf{一致连续}.
\end{definition}
\begin{theorem}
    设函数$f(x)$在区间$X$上定义,则$f(x)$在区间$X$上一致连续的充分必要条件是:对任何在$X$中的点列$\{x_n'\}$和$\{x_n''\}$,只要满足$\lim_{n\to\infty}(x_n'-x_n'')=0$,就成立
    \[\lim_{n\to\infty}(f(x_n')-f(x_n''))=0.\]
\end{theorem}
\begin{theorem}
    \textbf{(Cantor定理)}若函数$f(x)$在闭区间$[a,b]$上连续,则它在$[a,b]$上一致连续.
\end{theorem}

\newpage
\part{一元函数的微积分}
\section{微分}
\subsection{微分和导数}
\begin{definition}
    对函数$y=f(x)$定义域中的一点$x_0$,若存在一个只与$x_0$有关,而与$\Delta x$无关的数$g(x_0)$,使得当$x\to 0$时恒成立关系式
    \[\Delta y=f(x+\Delta x)-f(x)=g(x_0)\Delta x+o(\Delta x)\]
    则称$f(x)$在$x_0$处的微分存在,或称$f(x)$在$x_0$处可微.
\end{definition}
\par{显然,可微必然连续,但是连续不一定可微.因为连续函数只是极限存在,并不代表能表示成其线性主部的形式.}
\begin{definition}
    若函数$f(x)$在其定义域中的一点$x_0$处极限
    \[\lim_{\Delta x\to 0}\frac{\Delta y}{\Delta x}=\lim_{\Delta x\to 0}\frac{f(x_0+\Delta x)-f(x_0)}{\Delta x}\]
    存在,则称$f(x)$在$x_0$处可导,并称这个极限值为$f(x)$在$x_0$处的导数,记为$f'(x_0)$.\\
    若函数$y=f(x)$在某一区间上的每一点都可导,则称$f(x)$在该区间上可导.
\end{definition}
\par{事实上而言,一元函数在任一点的可微性和可导性是等价的.}
\subsection{导数的意义和性质}
\par{这节没啥干货,只是类似单侧极限,同样可以定义单侧导数,此处从略.}
\subsection{导数四则运算和反函数求导法则}
\par{导数四则运算:加减法显然,乘法如下
\[[f(x)g(x)]'=f'(x)g(x)+f(x)g'(x)\]
除法如下
\[[\frac{f(x)}{g(x)}]'=\frac{f'(x)g(x)-f(x)g'(x)}{[g(x)]^2}\]
}
\par{由数学归纳法,我们可以得到
\[[\prod_{i=1}^n f_i(x)]'=\sum_{j=1}^{n}\{f_j'(x)\prod_{i=1,\ i\neq j}^nf_i(x)\}\]}
\begin{theorem}
    \textbf{(反函数求导定理)}若函数$y=f(x)$在$(a,b)$上连续、严格单调、可导并且$f'(x)\neq 0$,记$\alpha=\min(f(a+),f(b-))$,$\beta=\max(f(a+),f(b-))$,则它的反函数$x=f^{-1}(y)$在$(\alpha,\beta)$上可导,且有
    \[[f^{-1}(y)]'=\frac{1}{f'(x)}\]
\end{theorem}
\par{常用导数微分表见下页.}
\begin{figure}[htbp]
    \centering
    \includegraphics[scale=0.65]{weifenbiao01.png}
    \includegraphics[scale=0.65]{weifenbiao02.png}
    \caption{常用导数微分表}
\end{figure}
\newpage
\subsection{复合函数求导法则及其应用}
\begin{theorem}
    \textbf{(复合函数求导法则)}设函数$u=g(x)$在$x=x_0$处可导,而函数$f(u)$在$u=u_0=g(x_0)$处可导,则复合函数$y=f(g(x))$在$x=x_0$可导,且有
    \[[f(g(x))]'|_{x=x_0}=f'(u_0)g'(x_0)=f'(g(x_0))g'(x_0).\]
\end{theorem}
\subsection{高阶导数和高阶微分}
\begin{theorem}
    \textbf{(Leibniz公式)}设$f(x)$和$g(x)$都是$n$阶可导函数,则它们的积函数也$n$阶可导,且
    \[[f(x)g(x)]^{(n)}=\sum_{k=0}^n C_n^kf^{(n-k)}(x)g^{(k)}(x).\]
\end{theorem}
\section{微分中值定理及其应用}
\subsection{微分中值定理}
\begin{definition}
    设$f(x)$在$(a,b)$上有定义,$x_0\in(a,b)$,如果存在点$x_0$的某一个邻域$O(x_0,\delta)\subset(a,b)$,使得\[f(x)\leq f(x_0),\quad x\in O(x_0,\delta)\]
    则称$x_0$是$f(x)$的一个极大值点,$f(x_0)$称为相应的极大值.同样可以定义极小值点和极小值.
\end{definition}

\begin{theorem}
    \textbf{(Fermat引理)}设$x_0$是$f(x)$的一个极值点,$f(x)$在$x_0$处导数存在,则$f'(x_0)=0$.
\end{theorem}

\begin{theorem}
    \textbf{(Rolle定理)}设函数$f(x)$在闭区间$[a,b]$上连续,在开区间$(a,b)$上可导,且$f(a)=f(b)$,则至少存在一点$\xi\in(a,b)$,使得$f'(\xi)=0$.
\end{theorem}

\begin{theorem}
    \textbf{(Lagrange中值定理)}设函数$f(x)$在闭区间$[a,b]$上连续,在开区间$(a,b)$上可导,则至少存在一点$\xi\in(a,b)$,使得
    \[f'(\xi)=\frac{f(b)-f(a)}{b-a}.\]
\end{theorem}

\begin{definition}
    设函数在区间$I$上定义,若对于$I$中的任意两点$x_1$和$x_2$,和任意的$\lambda\in(0,1)$,都有
    \[f(\lambda x_1+(1-\lambda)x_2\leq \lambda f(x_1)+(1-\lambda)f(x_2))\]
    则称$f(x)$是$I$上的下凸函数.
\end{definition}

\begin{theorem}
    设函数$f(x)$在区间$I$上二阶可导,则$f(x)$在区间$I$上是下凸函数的充分必要条件是:对于任意$x\in I$有$f''(x)\geq 0$.(特别地,如果恒大于零即为严格下凸函数)
\end{theorem}

\begin{theorem}
    \textbf{(Jensen不等式)}若$f(x)$为区间$I$的下凸函数,则对于任意$x_i\in I$和满足$\sum_{i=1}^n\lambda_i=1$的$\lambda_i>0$,成立
    \[f(\sum_{i=1}^n\lambda_ix_i)\leq\sum_{i=1}^n\lambda_if(x_i)\]
    特别地,取$\lambda_i=\frac{1}{n}$,就有
    \[f(\frac{1}{n}\sum_{i=1}^nx_i)\leq\frac{1}{n}\sum_{i=1}^nf(x_i)\]
    上凸函数同理.
\end{theorem}

\begin{theorem}
    \textbf{(Cauchy中值定理)}设$f(x)$和$g(x)$都在闭区间$[a,b]$上连续,在开区间$(a,b)$上可导,且对于任意$x\in(a,b)$,$g'(x)\neq 0$,则至少存在一点$\xi\in(a,b)$,使得
    \[\frac{f'(\xi)}{g(\xi)}=\frac{f(b)-f(a)}{g(b)-g(a)}\]
\end{theorem}

\subsection{L'Hospital法则}

\begin{theorem}
    \textbf{(L'Hospital法则)}设函数$f(x)$和$g(x)$在$(a,a+d]$上可导($d$是某个正常数),且$g'(x)\neq 0$. 若此时有
    \[\lim_{x\to a+}f(x)=\lim_{x\to a+}g(x)=0\]
    或
    \[\lim_{x\to a+}g(x)=\infty\]
    且$\lim_{x\to a+}\frac{f'(x)}{g'(x)}$存在(可以是有限数或$\infty$),则成立
    \[\lim_{x\to a+}\frac{f'(x)}{g'(x)}=\lim_{x\to a+}\frac{f(x)}{g(x)}\]
\end{theorem}
\subsection{Taylor公式和插值多项式}
\begin{definition}
    \textbf{(带Peano余项的Taylor公式)}设$f(x)$在$x_0$处有$n$阶导数,则存在$x_0$的一个邻域,对于该邻域中的任一点$x$,成立
    \[f(x)=f(x_0)+f'(x_0)(x-x_0)+\frac{f''(x)}{2!}(x-x_0)^2+...+\frac{f^{(n)}}{n!}(x-x_0)^n+r_n(x)\]
    其中余项$r_n(x)$满足
    \[r_n(x)=o({(x-x_0)^n}).\]
\end{definition}

\begin{definition}
    \textbf{(带Lagrange余项的Taylor公式)}设$f(x)$在$[a,b]$上具有$n$阶连续导数,且在$(a,b)$上有$n+1$阶导数.设$x_0\in[a,b]$为一定点,则对于任意$x\in[a,b]$,成立
    \[f(x)=f(x_0)+f'(x_0)(x-x_0)+\frac{f''(x)}{2!}(x-x_0)^2+...+\frac{f^{(n)}}{n!}(x-x_0)^n+r_n(x)\]
    其中余项$r_n(x)$满足
    \[r_n(x)=\frac{f^{(n+1)}(\xi)}{(n+1)!}(x-x_0)^{n+1}\]
    其中$\xi$在$x$和$x_0$之间.
\end{definition}
\par{插值多项式和数值计算部分略去,详情参考陈纪修书相应章节.}

\section{不定积分}
\subsection{不定积分的概念和运算法则}
\begin{definition}
    若在某个区间上,函数$F(x)$和$f(x)$成立关系
    \[F'(x)=f(x)\]
    或等价地,
    \[d(F(x))=f(x)dx\]
    则称$F(x)$是$f(x)$在这个区间上的一个原函数. 并且我们将一个函数$f(x)$的原函数全体称为这个函数的不定积分,记作$\int f(x)dx$.
\end{definition}
\subsection{换元积分和分部积分}
\paragraph{第一类换元积分法(凑微分法)}
\[\int f(x)dx=\int \tilde{f}(g(x))g'(x)dx=\int \tilde{f}(u)du=\tilde{F}(u)+C\]
\paragraph{第二类换元积分法(凑微分法)}
\[\int f(x)dx=\int \tilde{f}(\varphi(t))d\varphi(t)=\int \tilde{f}(\varphi(t))\varphi'(t)dt=\tilde{F}(t)+C\]
\paragraph{分部积分法}
\[\int u(x)v'(x)dx=u(x)v(x)-\int u'(x)v(x)dx\]
\subsection{有理函数的不定积分及其应用}
\begin{theorem}
    设有理函数$\frac{p(x)}{q(x)}$是真分式,多项式$q(x)$有$k$重实根$\alpha$,即$q(x)=(x-\alpha)^kq_1(x)$,$q_1(\alpha)=0$. 则存在实数$\lambda$与多项式$p_1(x)$,$p_1(x)$的次数低于$(x-\alpha)^{k-1}q_1(x)$的次数,成立
    \[\frac{p(x)}{q(x)}=\frac{\lambda}{(x-\alpha)^k}+\frac{p_1(x)}{(x-\alpha)^{k-1}q_1(x)}\]
\end{theorem}

\begin{theorem}
    设有理函数$\frac{p(x)}{q(x)}$是真分式,多项式$q(x)$有$l$重共轭复根$\beta+i\gamma$,即$q(x)=(x^2-2\beta x+\beta^2+\gamma^2)^l q^*(x)$,$q^*(\beta\pm i\gamma)\neq 0$. 则存在实数$\mu$,$\nu$和多项式$p^*(x)$,$p^*(x)$的次数低于$(x^2-2\beta x+\beta^2+\gamma^2)^{l-1} q^*(x)$的次数,成立
    \[\frac{p(x)}{q(x)}=\frac{\mu x+\nu}{(x^2-2\beta x+\beta^2+\gamma^2)^l}+\frac{p^*(x)}{(x^2-2\beta x+\beta^2+\gamma^2)^{l-1} q^*(x)}\]
\end{theorem}
\par{常用积分表见下页:}
\begin{figure}[htbp]
    \centering
    \includegraphics[scale=0.70]{jifenbiao.png}
    \caption{常用积分表}
\end{figure}
\newpage
\subsection{不定积分小妙招}
\section{定积分}
\subsection{定积分的概念和可积条件}
\subsection{定积分的基本性质}
\subsection{微积分基本定理}
\subsection{定积分在几何计算中的应用}
\section{反常积分}
\subsection{反常积分的概念和计算}
\subsection{反常积分的收敛判别法}



















\newpage
\part{级数}
\newpage
\part{多元函数的微积分}
\newpage
\part{Fourier级数}
\end{document}