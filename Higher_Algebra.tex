\documentclass[9pt]{ctexart}
\CTEXsetup[format={\Large\bf}]{section}
\usepackage{lmodern}
\usepackage{geometry}
\usepackage{titlesec}
\usepackage{amsmath}
\usepackage{fancyhdr}
\pagestyle{plain}
\usepackage{amssymb}
\usepackage{bm}
\usepackage{graphicx}
\usepackage{braket}
\usepackage{hyperref}
\usepackage{ntheorem}
\theorembodyfont{\normalfont}

\newtheorem{definition}{Definition}[section]
\newtheorem{theorem}{Theorem}[section]
\title{\bf{高等代数:提纲与注解}}
\author{\textsc{Frederick Wang}}
\date{}
\begin{document}
\maketitle
\section{行列式}
\subsection{$n$阶行列式}
\begin{definition}
    定义元素$a_{ij}$的余子式$M_{ij}$为行列式$|\bm{A}|$中划去第$i$行第$j$列后剩下的$n-1$行与$n-1$列元素组成的行列式.
\end{definition}
\begin{definition}
    定义$n$阶行列式的值为
    \[|\bm{A}|=a_{11}M_{11}-a_{21}M_{21}+...+(-1)^{n+1}a_{n1}M_{n1}\].
\end{definition}
\begin{definition}
    在行列式$|\bm{A}|$中,$a_{ij}$的代数余子式定义为$A_{ij}=(-1)^{i+j}M_{ij}$,其中$M_{ij}$是$A_{ij}$的余子式.
\end{definition}
\begin{theorem}
    行列式的性质如下
    \begin{itemize}
        \item[(1)] 上三角和下三角矩阵的行列式等于主对角线上元素之积.
        \item[(2)] 若$n$阶行列式$|\bm{A}|$的某一行或某一列的元素全为0,则$|\bm{A}|=0$.
        \item[(3)] 将行列式$|\bm{A}|$的某一行或某一列乘以常数$c$,则得到的行列式$|\bm{B}|=c|\bm{A}|$.
        \item[(4)] 对换行列式不同的两行,则行列式的值改变符号.(绝对值不变)
        \item[(5)] 若行列式的两行成比例,则$|\bm{A}|=0$. 特别地,如果两行相同,则行列式的值为0.
        \item[(6)] 设$|\bm{A}|$,$|\bm{B}|$,$|\bm{C}|$是三个$n$阶行列式,它们的第$(i,j)$元素分别记为$a_{ij}$,$b_{ij}$, $c_{ij}$。则它们三个行列式的第$r$行元素适合条件:\[c_{rj}=a_{rj}+b_{rj}\]而其它元素相同,则$|\bm{C}|=|\bm{A}|+|\bm{B}|$.
        \item[(7)] 将行列式的一行乘以某个常数$c$加到另一行上,行列式的值不变.
        \item[*]上述性质(4)到性质(7)对于列同样成立.
    \end{itemize}
\end{theorem}
\subsection{行列式的展开与转置}
\begin{theorem}
    设$|\bm{A}|$是$n$阶行列式,第$i$行第$j$列的元素$a_{ij}$的代数余子式为$A_{ij}$,则对任意的$r$,有展开式
    \[|\bm{A}|=a_{1r}A_{1r}+...+a_{nr}A_{nr}\]
    又对任意的$s\neq r$,有
    \[a_{1r}A_{1s}+...+a_{nr}A_{ns}=0.\]
    同样地,这个性质对于对行也同样成立.
\end{theorem}
\begin{theorem}
    行列式转置后的值不变,即$|\bm{A}'|=|\bm{A}|$.
\end{theorem}
\begin{definition}
    \textbf{(Cramer法则)}设有线性方程组$\bm{A}x=\bm{b}$,若$|\bm{A}|\neq 0$,则方程有且只有一组解:
    \[x_i=\frac{|\bm{A}_i|}{|\bm{A}|}\]
    其中$|\bm{A}_i|$是一个$n$阶行列式,它由$|\bm{A}|$去掉第$i$列换上列向量$\bm{b}$组成.
\end{definition}
\subsection{行列式的计算}
\paragraph{Vande Monde行列式}
\begin{equation*}
    V_n=
    \left(
    \begin{matrix}
        1&x_1&...&x_1^{n-1}\\
        1&x_2&...&x_2^{n-1}\\
        ...&...&...&...\\
        1&x_n&...&x_n^{n-1}
    \end{matrix}
    \right)
\end{equation*}
\[|V_n|=\prod_{1\leq i< j\leq n}(x_j-x_i)\]
\subsection{行列式的等价定义}

\subsection{Laplace定理}











\section{矩阵}
\subsection{矩阵的概念}
\subsection{矩阵的运算}
\subsection{方阵的逆阵}
\subsection{矩阵的初等变换与初等矩阵}
\subsection{矩阵乘积的行列式与初等变换法求矩阵}
\subsection{分块矩阵}
\subsection{Cauchy-Binet公式}
\section{线性空间}
\end{document}